% !TEX TS-program = pdflatex
% !TEX encoding = UTF-8 Unicode

% This is a simple template for a LaTeX document using the "article" class.
% See "book", "report", "letter" for other types of document.

\documentclass[11pt]{article} % use larger type; default would be 10pt

\usepackage[utf8]{inputenc} % set input encoding (not needed with XeLaTeX)

%%% Examples of Article customizations
% These packages are optional, depending whether you want the features they provide.
% See the LaTeX Companion or other references for full information.

%%% PAGE DIMENSIONS
\usepackage{geometry} % to change the page dimensions
\geometry{a4paper} % or letterpaper (US) or a5paper or....
% \geometry{margins=2in} % for example, change the margins to 2 inches all round
% \geometry{landscape} % set up the page for landscape
%   read geometry.pdf for detailed page layout information

\usepackage{graphicx} % support the \includegraphics command and options
\usepackage{listings}
\usepackage{color} %red, green, blue, yellow, cyan, magenta, black, white
\definecolor{mygreen}{RGB}{28,172,0} % color values Red, Green, Blue
\definecolor{mylilas}{RGB}{170,55,241}
\usepackage{amsmath, amsthm, amsfonts}
\usepackage{titlesec}
\usepackage{amsmath}
\usepackage{amssymb}
\usepackage{amsthm}
\usepackage{latexsym}
\usepackage{mathrsfs}
\usepackage{graphicx}
\usepackage{arcs}

% \usepackage[parfill]{parskip} % Activate to begin paragraphs with an empty line rather than an indent

%%% PACKAGES
\usepackage{kotex}
\DeclareGraphicsExtensions{.pdf,.png,.jpg}
\usepackage{booktabs} % for much better looking tables
\usepackage{array} % for better arrays (eg matrices) in maths
\usepackage{paralist} % very flexible & customisable lists (eg. enumerate/itemize, etc.)
\usepackage{verbatim} % adds environment for commenting out blocks of text & for better verbatim
\usepackage{subfig} % make it possible to include more than one captioned figure/table in a single float
% These packages are all incorporated in the memoir class to one degree or another...

%%% HEADERS & FOOTERS
\usepackage{fancyhdr} % This should be set AFTER setting up the page geometry
\pagestyle{fancy} % options: empty , plain , fancy
\renewcommand{\headrulewidth}{0pt} % customise the layout...
\lhead{}\chead{}\rhead{}
\lfoot{}\cfoot{\thepage}\rfoot{}

%%% SECTION TITLE APPEARANCE
\usepackage{sectsty}
\allsectionsfont{\sffamily\mdseries\upshape} % (See the fntguide.pdf for font help)
% (This matches ConTeXt defaults)

%%% ToC (table of contents) APPEARANCE
\usepackage[nottoc,notlof,notlot]{tocbibind} % Put the bibliography in the ToC
\usepackage[titles,subfigure]{tocloft} % Alter the style of the Table of Contents
\usepackage{enumitem}
\renewcommand{\cftsecfont}{\rmfamily\mdseries\upshape}
\renewcommand{\cftsecpagefont}{\rmfamily\mdseries\upshape} % No bold!
\newtheorem{prop}{Proposition}
\newtheorem{thm}{Theorem}
\newtheorem{axiom}{Axiom}
\usepackage{cancel}
\newtheorem{p}{Problem}
\newtheorem*{lem}{Lemma}
\theoremstyle{definition}
\newtheorem{dfn}[thm]{Definition}
\newtheorem{rem}[thm]{Remark}
\newtheorem{exa}[thm]{Examples}
 \newenvironment{s}{%\small%
        \begin{trivlist} \item \textbf{Solution}. }{%
            \hspace*{\fill} $\blacksquare$\end{trivlist}}%
%%% END Article customizations

%%% The "real" document content comes below...

\title{HW2, Dept: 수리과학과, NAME: 국윤범}
\date{} % Activate to display a given date or no date (if empty),
         % otherwise the current date is printed 

\begin{document}
\maketitle

\lstset{language=Matlab,%
    %basicstyle=\color{red},
    morekeywords={matlab2tikz},
    keywordstyle=\color{blue},%
    morekeywords=[2]{1}, keywordstyle=[2]{\color{black}},
    identifierstyle=\color{black},%
    stringstyle=\color{mylilas},
    commentstyle=\color{mygreen},%
    showstringspaces=false,%without this there will be a symbol in the places where there is a space
    numbers=left,%
    numberstyle={\tiny \color{black}},% size of the numbers
    numbersep=9pt, % this defines how far the numbers are from the text
    emph=[1]{for,end,break},emphstyle=[1]\color{red}, %some words to emphasise
    %emph=[2]{word1,word2}, emphstyle=[2]{style},    
}

\section*{Problem 1 - Exercise 4.3}

\textbf{Problem 1 main script}
\lstinputlisting{hw2_1.m}
\textbf{Problem 1 function script}
\lstinputlisting{transformation.m}

The following is the result of the codes above. (m3, m41, m42, m43, m44 in order)

\begin{figure}[htbp]
\begin{center}
    \includegraphics[scale=0.6]{hw2_1a}
\end{center}
\begin{center}
    \includegraphics[scale=0.6]{hw2_1b}
\end{center}
\begin{center}
    \includegraphics[scale=0.6]{hw2_1c}
\end{center}
\end{figure}
\newpage

\begin{figure}
\begin{center}
    \includegraphics[scale=0.6]{hw2_1d}
\end{center}
\begin{center}
    \includegraphics[scale=0.6]{hw2_1e}
\end{center}
\end{figure}


\section*{Problem 2 - Exercise 9.3}
\lstinputlisting{hw2_2.m}

(a) The result of $spy("HELLO"\ matrix)$
\begin{center} 
\includegraphics[scale=0.4]{2a}
\end{center}

(b)
(The list of singular values of the matrix)
\begin{center}
\includegraphics[scale=0.5]{2b0}
\end{center}

(plot and semilogy)
\begin{center}
\includegraphics[scale=0.3]{2b}
\end{center}
\newpage

Mathematically, the exact rank of matrix is 10. Note that each letter has 2 independent column vectors and the last position of `1' in each column vectors in each letter differs by 1. Hence, $2\times 5 = 10$ independent column vectors span the column space.

In fact, we can check out in the first picture under (b) that $rank(mat)$ yields $10$. Also, when counting nonzero singular values in the picture, $0.6329$ is the last nonzero singular values, so that $10$ nonzero singular values exist in total. 
\\

(c) The following is low rank approximation from $1$ to $10$.
\begin{center}
\includegraphics[scale=0.25]{2b1}
\end{center}
\begin{center}
\includegraphics[scale=0.25]{2b2}
\end{center}

\section*{Problem 3}
\lstinputlisting{hw2_3.m}

(a)
\begin{center}
\includegraphics[scale=0.4]{3a}
\end{center}

(b)
\begin{center}
\includegraphics[scale=0.7]{3b}
\end{center}
Since the rank of the matrix is $3$, only 3 left and right singular vectors with 3 singular values are enough to recover the original matrix. Hence, the smallest number of entries to store as data is computed like $rank(matrix)\cdot (1 + 60 + 60)$, where $60$ is the length of singular vectors and $1$ corresponds to singular values.\\
\newpage
(c)
\begin{center}
\includegraphics[scale=0.4]{3c}
\end{center}

Comparing (c) with (a), we can conclude that the method in (b) recover the original data without loss of data.

\section*{Problem 4}
\lstinputlisting{hw2_4.m}

(a)
\begin{center}
\includegraphics[scale=0.75]{4a}
\end{center}

(b)
\begin{center}
\includegraphics[scale=0.7]{4b}
\end{center}


(c)
\begin{center}
\includegraphics[scale=0.7]{4c}
\end{center}



\end{document}